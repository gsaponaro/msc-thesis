%!TEX encoding = UTF-8 

\selectlanguage{italian}%
\begin{abstract}
La robotica umanoide in generale, e l'interazione uomo--robot in particolare, stanno oggigiorno guadagnando nuovi e vasti campi applicativi: la robotica si diffonde sempre di pi\`u nella nostra vita. Una delle azioni che i robot umanoidi devono poter eseguire \`e la manipolazione di cose (avvicinare le braccia agli oggetti, afferrarli e spostarli). Tuttavia, per poter fare ci\`o un robot deve prima di tutto possedere della \emph{conoscenza} sull'oggetto da manipolare e sulla sua posizione nello spazio. Questo aspetto si pu\`o realizzare con un approccio percettivo.

Il sistema sviluppato in questo lavoro di tesi \`e basato sul \emph{tracker} visuale CAMSHIFT e su una tecnica di ricostruzione 3D che fornisce informazioni su posizione e orientamento di un oggetto generico (senza modelli geometrici) che si muove nel campo visivo di una piattaforma robotica umanoide. Un oggetto \`e percepito in maniera \emph{semplificata}: viene approssimato come l'ellisse che racchiude meglio l'oggetto stesso.

Una volta calcolata la posizione corrente di un oggetto situato di fronte al robot, \`e possibile realizzare il \emph{reaching} (avvicinamento del braccio all'oggetto). In questa tesi vengono discussi esperimenti ottenuti col braccio robotico della piattaforma di sviluppo adottata.
\end{abstract}

\selectlanguage{english}%
\begin{abstract}
Humanoid robotics in general, and human--robot interaction in particular, is gaining new, extensive fields of application, as it gradually becomes pervasive in our daily life. One of the actions that humanoid robots must perform is the manipulation of things (reaching their arms for objects, grasping and moving them). However, in order to do this, a robot must first have acquired some \emph{knowledge} about the target object and its position in space. This can be accomplished with a perceptual approach.

The developed system described in this thesis is based on the CAMSHIFT visual tracker and on a 3D reconstruction technique, providing information about position and orientation of a generic, model-free object that moves in the field of view of a humanoid robot platform. An object is perceived in a \emph{simplified} way, by approximating it with its best-fit enclosing ellipse.

After having computed where an object is currently placed in front of it, the robotic platform can perform reaching tasks. Experiments obtained with the robot arm of the adopted platform are discussed.
\end{abstract}
